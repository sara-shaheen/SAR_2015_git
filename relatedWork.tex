In this section, we survey previous work that is most related to our approach, grouped into four main categories.

%\vspace{-4mm}
\noindent\textbf{Shape Matching.} \label{subsec: shapematching}
%\vspace{-1mm}
Determining stroke authorship seems akin to shape matching, but has different requirements as further inspection is needed. Numerous methods have been developed for shape matching and classification in the past ~\cite{mokhtarian1992theory,belongie2001matching,jin2003image,berg2005shape}, as well as, recently \cite{Michel:2011:SID:1994006.1994152,ion-cviu-11}. Shape matching searches for similar shapes between two images, where one is usually considered the query image. Our system differs from shape matching in two main aspects. First, shape matching focuses on global information of contours such as zero crossings of curvature, while SAR segments the contour and studies detailed local features from each curve segment e.g. the eccentricity of a conic fit. Second, shape matching is usually applied to low resolution images for computational reasons, where these images tend to be classified by their content, e.g., a set of different mice silhouettes/shapes from different artists tend to belong to the same object class. Our SAR technique can be employed efficiently on images of various sizes, especially those at high resolution. Unlike shape matching, SAR is less dependent on sketch content and more focused on the local intricacies of sketch style. %For example, SAR can predict who drew a particular 'flower' sketch even though it is

%\vspace{-4mm}
\noindent\textbf{Sketch and Artistic Style Analysis.}\label{subsec: artisticanalysis}
%\vspace{-1mm}
A data driven approach in \cite{Berger:2013:SAP:2461912.2461964} is developed to analyze style in portrait sketches. This technique is also used for portrait sketch synthesis. While their focus is to mimic and synthesize a particular sketching style of only human portrait sketches, our focus is to discriminate sketches based on their authorship (i.e. artistic style). In their study, they analyze sketches at the level of strokes and shapes. However, they handle stroke analysis differently, since they focus on global features. SAR on the other hand exploits local stroke features that are necessary in discriminating authorship of sketches that look similar as is the case in sketch fraud detection. Moreover, in \cite{Berger:2013:SAP:2461912.2461964}, portrait sketches were digitally collected using the Wacom pen to build a library of strokes while we allow artists to express their style freely and to erase or redraw their sketches using any medium they prefer. Interestingly, we use the sketch synthesis dataset of \cite{Berger:2013:SAP:2461912.2461964} to demonstrate how SAR can quantitatively evaluate sketch synthesis. SAR results are on par with those reported by the authors after an online study with human participants.

In \cite{Limpaecher:2013:RDA:2461912.2462016}, a total of 13,000 drawings of faces was collected. Unlike our work, they do not study the problem of authorship as they focus on auto-correction of strokes for novice artists. In \cite{Lu:2012:HES}, the authors mimicked a particular artistic style via matching that was based on filtered velocities and shape context. Concurrently, the work of \cite{Kalogerakis:2012:mlhatching} synthesized new drawings using hatching styles of sketching that are learned by example, while that of \shortcite{Freeman03learningstyle} provided an example-based method to modify line drawings for the purpose of reproducing different artistic styles. Also, the approach of \shortcite{Cole:2008:PDL:1360612.1360687} studied where artists draw lines in sketches of objects such as tools, automobile parts and bones. They concluded that artists tend to draw similar lines in consistent locations. In our experiments, we use the dataset of this work to show interesting new results regarding the uniqueness of sketch style despite strict sketching constraints imposed on the artists. Finally, a large body of work focuses on generating artistically stylized rendering using 2D input images or videos of non-photorealistic rendering (NPR). We refer the reader to the survey of \cite{Kyprianidis:2013:TAS} for details. Although this work targets the analysis of sketches and artistic styles of different sketches, it does not address the important problem of how authorship can be determined based on stroke cues manifesting themselves in sketches. %similar to our method in terms of analyzing similar sketches and conclude authorship of those sketches based on a classification model.


%\vspace{-4mm}
\noindent\textbf{Sketch Recognition and Retrieval.}
%\vspace{-1mm}
The work of \shortcite{eitz2012hdhso} developed an automated data-driven method to explore a large collection of hand-drawn sketches using drawings collected by many non-experts. Their primary goal was to represent sketch content to perform object recognition from a sketch and \emph{not} sketch style. Concurrently, in~\cite{Sun:2012:SAH:2393347.2396429}, a system was proposed for real-time recognition and retrieval of semantically meaningful attributes of hand-drawn sketches. Their work is not limited to pre-defined object classes. Other sketch retrieval methods are based only on geometric similarity between sketches ~\cite{Shrivastava:2011:DVS:2024156.2024188,5674030}. Unlike our work, the field of sketch recognition and retrieval is based on classifying sketch content from a discrete number of semantic categories. They do not provide comparisons among similar sketches or across artistic styles and do not address the problem of authorship recognition.
%As such, they do not address the problem of authorship recognition.
%SThose methods do not learn by example which makes sketch retrieving an efficient nearest-neighbor problem but they lack the semantic understanding of sketches

%Removed ref: \cite{Chalechale05sketch-basedimage}
%\vspace{-4mm}
\noindent\textbf{Forensic Handwriting Analysis.}
%\vspace{-1mm}
Forensic handwriting analysis is a well studied problem \cite{srisurvey03} as is signature analysis, and more precisely, off-line feature based signature verification \cite{4603099,Kovari2013247}. Handwriting analysis tools tend to be very specific to the problem domain, so they use features centric to handwriting such as letter height, pitch, baselines, crossings, etc. They do not generalize well to sketch analysis \cite{1263248}. Their focus is on the uniqueness of letter and punctuation formation, flow and structure \cite{handwriting2001}, whereas we investigate the uniqueness of hand-drawn strokes in a more general context.

%Forensic handwriting analysis is a well studied problem and entails a large body of previous work \cite{srisurvey03}. The aim here is to develop techniques that identify authorship by comparing hand-written samples of different people. Similarly, signature analysis and more precisely off-line feature based signature verification has also been studied for many years \cite{4603099}. More recent work can be found in \cite{Kovari2013247} and \cite{signature2013}. While all this work strives to prove the individuality of only handwriting and signatures \cite{srihari_2002_jfs} \cite{handwriting2001}, we investigate the uniqueness of hand-drawn strokes in general. As a result, features extracted in handwriting analysis are only applicable to that specific domain (e.g., height of a loop, pitch, baselines etc.) and not generalizable to sketch analysis \cite{1263248}. On the other hand, SAR provides a general framework for stroke representation using second-order curvature analysis and the Bag-of-Words (BoW) model, which to our best knowledge has not been explored in the area of handwriting analysis. %We believe that such representation was not utilized in the area of handwriting analysis and signature verification.

% removed refs: \cite{srihari_2002_jfs} \cite{signature2013}
%\vspace{-4mm}
\noindent\textbf{Graphical Based User Authentication.}
%\vspace{-1mm}
A number of graphical based user authentication methods such as doodle sketches were proposed as an alternative to conventional authentication methods. As an example, PassDoodle is a graphical based authentication mechanism, which attempts to identify users by their handwritten designs (doodles). For authentication, the query doodle is represented on a regular grid and matched to training doodles in order to determine user authenticity \cite{varenhorst2004passdoodles,Govindarajulu:2007:PMU:1322192.1322233}. The effectiveness of using PassDoodles for user authentication is demonstrated here \shortcite{Renaud20091}. In \shortcite{4761233}, on the other hand, a sketch-based authentication method was developed to find the \emph{closest} sketch in a training database by extracting edge orientation features from a user's query sketch. Unlike SAR, these authentication methods neither build a classification model nor provide an analysis of different artistic styles. Moreover, they do not build an intermediate sketch representation of the user input which makes them similar to shape matching techniques. An exposition of the shortcomings of graphical based authentication methods is given in \cite{gani2010new}.
%The features are determine based on the distribution of the doodle on a regular grid.


%as SAR and thus user input should be kept simple with minimum number of strokes. SAR's discrimination, on the other hand, is based on determining the frequency in which an artist uses certain basic strokes in comparison with other artists which makes it independent of the complexity of the sketch input.



