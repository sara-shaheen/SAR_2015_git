% ---------------------------------------------------------------------------
% Author guideline and sample document for EG publication using LaTeX2e input
% D.Fellner, v1.13, Jul 31, 2008

\documentclass{egpubl}
\usepackage{eg2015}

% --- for  Annual CONFERENCE
\ConferenceSubmission   % uncomment for Conference submission
%\ConferencePaper        % uncomment for (final) Conference Paper
% \STAR                   % uncomment for STAR contribution
% \Tutorial               % uncomment for Tutorial contribution
% \ShortPresentation      % uncomment for (final) Short Conference Presentation
% \Areas                  % uncomment for Areas contribution
% \MedicalPrize           % uncomment for Medical Prize contribution
% \Education              % uncomment for Education contribution
%
% --- for  CGF Journal
% \JournalSubmission    % uncomment for submission to Computer Graphics Forum
% \JournalPaper         % uncomment for final version of Journal Paper
%
% --- for  CGF Journal: special issue
% \SpecialIssueSubmission    % uncomment for submission to Computer Graphics Forum, special issue
% \SpecialIssuePaper         % uncomment for final version of Journal Paper, special issue
%
% --- for  EG Workshop Proceedings
% \WsSubmission    % uncomment for submission to EG Workshop
% \WsPaper         % uncomment for final version of EG Workshop contribution
%
 \electronicVersion % can be used both for the printed and electronic version

% !! *please* don't change anything above
% !! unless you REALLY know what you are doing
% ------------------------------------------------------------------------

% for including postscript figures
% mind: package option 'draft' will replace PS figure by a filname within a frame
\ifpdf \usepackage[pdftex]{graphicx} \pdfcompresslevel=9
\else \usepackage[dvips]{graphicx} \fi


\PrintedOrElectronic

% prepare for electronic version of your document
% prepare for electronic version of your document
\usepackage{t1enc,dfadobe}
\usepackage{egweblnk}
\usepackage{cite}
\usepackage{epstopdf}
\usepackage{float}
\usepackage{verbatim}
\usepackage{subfigure}
\usepackage{amsmath}
\usepackage{multicol}
\usepackage{caption}
\usepackage{fontenc}
\usepackage{csquotes}
\usepackage{array}
\usepackage{graphicx}
% For backwards compatibility to old LaTeX type font selection.
% Uncomment if your document adheres to LaTeX2e recommendations.
% \let\rm=\rmfamily    \let\sf=\sffamily    \let\tt=\ttfamily
% \let\it=\itshape     \let\sl=\slshape     \let\sc=\scshape
% \let\bf=\bfseries

% end of prologue

% ------------------------------------------------------------------------

% if the Editors-in-Chief have given you the data, you may uncomment
% the following five lines and insert it here
%
% \volume{27}   % the volume in which the issue will be published;
% \issue{1}     % the issue number of the publication
% \pStartPage{1}      % set starting page


%-------------------------------------------------------------------------

% For backwards compatibility to old LaTeX type font selection.
% Uncomment if your document adheres to LaTeX2e recommendations.
% \let\rm=\rmfamily    \let\sf=\sffamily    \let\tt=\ttfamily
% \let\it=\itshape     \let\sl=\slshape     \let\sc=\scshape
% \let\bf=\bfseries

% end of prologue

%--------------------------------------------------------------
%--------------------------------------------------------------
\title[SAR: Stroke Authorship Recognition]%
      {SAR: Stroke Authorship Recognition}

% for anonymous conference submission please enter your SUBMISSION ID
% instead of the author's name (and leave the affiliation blank) !!
\author[paper1004]
     {paper1004}

% ------------------------------------------------------------------------

% if the Editors-in-Chief have given you the data, you may uncomment
% the following five lines and insert it here
%
% \volume{27}   % the volume in which the issue will be published;
% \issue{1}     % the issue number of the publication
% \pStartPage{1}      % set starting page


%-------------------------------------------------------------------------
\begin{document}


\teaser{
\centering
\includegraphics[width = 1.01\textwidth]{images/fullPipeline2.png}
\vspace{-5mm}\caption {The SAR pipeline: 1. compile sketches from different artists; 2. extract strokes and split them into segments; 3. characterize each stroke segment mathematically 4. represent each sketch as a distribution of feature frequency; 5. Train a machine to recognize authorship based on 4. {\color{red}[The first module should be 'Datasets of Drawings Created by Artists'. The third should read (if space allows) 'Mathematical Characterization of Strokes'.]}}%\vspace{-2mm}
\label{pipeline}}

\maketitle

\begin{abstract}
Are simple strokes unique to the artist or designer who renders them? If so, can this idea be used to identify authorship or to classify artistic drawings? Also, could training methods be devised to develop particular styles? To answer these questions, we propose the Stroke Authorship Recognition (SAR) approach, a novel method that distinguishes the authorship of 2D digitized drawings. SAR converts a drawing into a histogram of stroke attributes that is discriminative of authorship. We provide extensive classification experiments on a large variety of datasets, which validate SAR's ability to distinguish unique authorship of artists and designers. We also demonstrate the usefulness of SAR in several applications including the detection of  fraudulent sketches, the training and monitoring of artists in learning a particular new style, and the first quantitative way to measure the quality of automatic sketch synthesis tools.\vspace{-1mm}
\end{abstract}
%-------------------------------------------------------------------------
\vspace{-5mm}
\section{Introduction}
\vspace{-2mm}
\input{introduction}

\vspace{-3mm}
\section{Related Work}
\vspace{-2mm}
\input{relatedWork}

\vspace{-3mm}
\section{Stroke Authorship Recognition} \label{sec:SAR}
\vspace{-1mm}
\input{strokeAuthorshipRecognition}

\vspace{-3mm}
\section{Sketch Datasets}\label{sec:datasets}
\vspace{-2mm}
\input{multipleDatasetsCreation}

\vspace{-3mm}
\section{Human Sketch Authorship Recognition}\label{sec:humanexps}
\vspace{-2mm}
\input{humanStrokeAuthorshipRecognition}

\vspace{-3mm}
\section{Experimental Results and Evaluation}\label{sec:exp}
\vspace{-2mm}
\input{experimentalResultsandEvaluation}

\vspace{-3mm}
\section{Applications} \label{sec:app}
\vspace{-2mm}
\input{applications}

\vspace{-3mm}
\section{Conclusions and Future Work}\label{sec:conclusions}
\vspace{-1mm}
In this paper, we shed light on a new direction in sketch analysis, namely authorship recognition through stroke analysis. We propose a stroke authorship recognition (SAR) approach that discriminates between artistic sketch styles based on the choice and frequency of use of basic strokes. From our extensive experiments and user studies (Section \ref{subsec:recognition}), we provide empirical evidence that SAR \emph{does} encode unique and consistent characteristics of an artist's sketching style, which are in turn used to discriminate one artist's sketches from others. we also showed that the extent to which SAR can be used for discrimination is dependent on the sketching constraints imposed on the artist. Moreover, our extensive user studies empirically validate the difficulty of this fine-grained recognition problem and indicate that SAR can improve upon human performance. All the compiled data (datasets and user studies) and source code will be made publicly available to enable further research on this exciting topic and to allow for quantitative comparison with future methods.



%\emph{Uniqueness of Sketch Style.} Based on results in Section \ref{subsec:recognition}, we conclude that SAR \emph{does} encode unique and consistent characteristics of an artist's sketching style, which are in turn used to discriminate one artist's sketches from others. This result validates SAR's applicability in important real-world tasks, such as sketch fraud detection (e.g. for design patents and cartoon characters) and training/teaching artistic style.

%\emph{Style and Sketch Constraints.} The extent to which SAR can be used for discrimination is dependent on the sketching constraints imposed on the artist. Although overall SAR accuracy decreases with more constraints, unique elements of artistic style still persist even under the strictest of constraints (fraud).

%\emph{Silhouettes.} We identify that a sketch's silhouette is a richer source of discriminative information than internal strokes in general. %We validate the choice of our proposed segmentation method from a classification point-of-view, as well as, determine the optimal levels of digitization needed for accurate representation.

%\emph{Human Performance.} Our extensive user studies empirically validate the difficulty of this fine-grained recognition problem and indicate that SAR can improve upon human performance. All the compiled data (datasets and user studies) and source code will be made publicly available to enable further research on this exciting topic and to allow for quantitative comparison with future methods.


%\vspace{-3mm}
\textbf{Future Work} We aim to improve the discriminative nature of SAR by enhancing the learned universal stroke segment dictionary. One way to do this is to investigate how strokes are actually generated by the artists through tracking their hand movements while sketching. We believe this will provide us with more information about an artist's style of sketching. Although the segmentation process is currently viewed as a pre-processing step in SAR, we aim to investigate how supervised information (artist labels) can be incorporated in this process in order to produce stroke segments that are inherently discriminative. 

%We have presented a new method of shape representation and authorship classification. It is based on analyzing local features of the sketch's silhouettes and internal strokes which are then used to represent a sketch as a histogram of universal strokes segments. The later representation is used in a k-NN based authorship classification model. The proposed method is applicable to sketches in various sizes, and invariant to non-rigid motion, scaling and rotation. Experimental results on a number of datasets we compile in this work demonstrate that our method has a good performance on authorship discrimination and classification.




%Towards the end of this paper we demonstrate a couple of applications which are built using the best performing computational model. The first one is an artistic fraud detection application that is based on one of the datasets we are providing. The second application is a training program that allows artists, designers and animators to test their affinity to any given artist, whose works have been incorporated into the machine learning part of the program, it is also designed based on one of our datasets.


\bibliographystyle{eg-alpha}
%\bibliographystyle{eg-alpha-doi}
\bibliography{egbibsample}
\end{document}
