Research in voice and face recognition have well-developed methods to recognize and distinguish individuals based on data cues ~\cite{tolba2006face}. Similarly, we investigate whether individuals (specifically artists) are distinguishable by the way they draw, and if so, the extent to which this uniqueness can identify their drawings, to help train them, or to detect sketch fraud. To our knowledge, this work is the first on authorship recognition from drawings. Our new method, called sketch authorship recognition, or SAR assumes that an artist's style can be recognized by the frequency distribution of certain mathematical descriptors, which are deduced from basic sketching strokes. A fundamental assumption is that the artist style manifests itself mathematically in the artist's basic strokes, For example, we have seen that Disney's Mickey Mouse tends to exhibit rounded strokes using many, nearly elliptic curves, whereas Looney Tunes Daffy Duck consists of a combination of straighter and tightly curved strokes. By extracting enough strokes from a digitized drawing, a characteristic histogram is created. This histogram captures an inherent style of the overall drawing.  As illustrated in Figure 1, SAR consists of five steps: (1) obtain sketches drawn, (2), extract strokes from each sketch, (3) split strokes into simple segments, (4) collect pertinent mathematical attributes of strokes into a frequency distribution, and (5) use (4) to train a machine to recognize the author.

\vspace{-1mm}
SAR is useful in a variety of applications such as fraud detection (i.e. fraudulent versus original sketches), where there is a need to examine fine-grained stroke-level features to discriminate sketches that \emph{look} similar. This problem cannot be approached using existing shape matching techniques, which are unable to detect local and fine differences. Also, these methods are heavily dependent on the content of the sketch and not its stlye. Moreover, SAR can play a major role in artistic style training and reproduction, since it can be used to quantitatively assess how the sketching style of an artist in-training is becoming similar to a desired target style throughout the training process. For example, newly recruited artists at Disney
are required to undergo a six month training procedure to familiarize themselves with the company's sketch techniques. This is done to ensure that new and already existing characters can be created with the same Disney \emph{look}. Clearly, there a need for an automated technique to examine how an artist progresses during his/her sketch style training. Other areas that can make use of SAR include handwriting verification, design patent litigation and brand marking.

%Techniques used in SAR can also be extended to handwriting verification applications, since automatic techniques that analyze handwriting authorship using detailed low-level features do not exist \B{I dont know whether we should say this. Reviewers might expect to see results}.
\vspace{-1mm}
\textbf{Contributions:} In this work, \textbf{(1)} we propose a novel method of sketch authorship recognition (SAR) based on the hypothesis that the collection of an artist's strokes in a sketch are unique to that artist and they can be used to define his/her style which can be used to detect sketch fraud. \textbf{(2)} We compile 2 new sketch datasets collected from  a number of experienced artists. These datasets are designed to expose SAR to different levels of challenges and sketch variations. They will be made publicly available, along with SAR source code, to allow for further research on this topic. \textbf{(3)} We develop two SAR-enabled sketch applications. The first provides artists in-training with immediate feedback on how close their sketching style is to a particular target style and monitors their progress throughout the training period. The other provides the first quantitative and automatic measure to evaluate the quality of automatic sketch synthesis tools.

%The paper is organized as follows. First, we survey the literature most related to SAR. Next, Section \ref{sec:SAR} presents a detailed description of SAR and its underlying modules. In Section \ref{sec:humanexps}, we present the details of the different datasets compiled in this work as well as 2 extensive user studies that validate the difficulty of the stroke authorship problem. Extensive experimental results and two sample applications of SAR are presented in Sections \ref{sec:exp} and \ref{sec:app} respectively. We conclude the paper and highlight future work in Section \ref{sec:conclusions}.


%As illustrated in Figure ~\ref{pipeline}, the proposed SAR approach consists of five major steps: obtaining sketches drawn by different artists, extracting strokes from each sketch, splitting strokes into segments that expose authorship, examining the characteristics of these segments to represent each sketch using features derived from these characteristics, and classifying authorship using this representation. To evaluate SAR, we compile 2 different sketch datasets collected from  a number of experienced artists. These datasets are designed to expose SAR to different levels of challenges and sketch variations so as to test its effectiveness in authorship recognition. These datasets, along with SAR source code, will be made publicly available to allow for further research in this field. \B{is this the contribution paragraph? usually contributions are made explicit}

%We hope that people in the community will find our databsets useful to use and research upon. Our collaborating artists who demonstrated through their work a great sketching abilities, have come from diverse artistic and sketching backgrounds as some of them are graphical designers and others are interior designer with an average sketching experience of 7 years and a maximum of 10 years.




%conducted to compare human versus computational sketch authorship recognition performance will be provided. After that, the development pipeline of SAR and the different techniques adopted will be provided. Towards the end of this paper, we share a number experimental results across different datasets that validate the use of SAR in authorship recognition.We will also present experimental results regarding different algorithmic variations we adopted and tested so that the community can use our findings and take them into consideration when conducting a research in the same field. Finally, we will demonstrate an application which is a training program that allows artists, designers and animators to test their affinity to any given artist, whose works have been incorporated into the machine learning part of the program.

